%!TEX root = project.tex

\chapter*{About this project}
\paragraph{Abstract}
GMIT cross-platform mobile banking application is an app created using the Ionic version one framework which uses Javascript, HTML5 and Angular. This is a mock Banking App designed for GMIT students where they can view their recent transactions and transfer and manage money in their account. This application focuses on security and has AuthO implemented as the log in  

\paragraph{Authors}
Explain here who the authors are.



\chapter{Introduction}
What is mobile banking? Mobile banking is a service provided by a bank or other financial institution that allows its customers to conduct financial transactions remotely using a mobile device such as a mobile phone or tablet\cite{mobilebankingwiki}. It uses software, usually called an app, provided by the financial institution for the purpose\cite{mobilebankingwiki}. Mobile banking provides users an easier way of managing their financial affairs. It's faster, you can use it on the go and you also don't have to wait in long lines in banks or travel to your nearest ATM.

Our main objective of this project was to focus on security and also to create a cross platform application that we could use on both Android and IOS. We came up with the idea of a mock banking application for GMIT where students could have access to their account and manage their payments by transferring mock money and also looking at their recent transactions.

A banking application tackles a lot of the main aspects in programming in the form of user 
confidentiality. So we decided to take on a banking application covering all the main aspects in a banking application.
	Our Banking Application is a cross platform app which means it can be used on both 
Android and IOS phones and this is done by using the Ionic Framework. We decided to use Ionic v1
as we wanted to work with HTML5 , Javascript and Angular. We have a MongoDB set up so we can 
store and retrieve data and also protect the users information. As this is not an official 
banking app we decided to create a mock one where a new user can register and also manage their 
own personal accounts by transferring money to someone else's account where you can add a new 
Payee and transfer money to that Payee. 

Our idea at first was to aim our application at credit union customers but we then decided it would be a better idea to aim something at the students at GMIT so we set about creating a mobile banking application for students who attend GMIT. From here students would be able to manage their finances on the go and have their own savings account. They can even transfer money to other students across the college in the click of a button. With a simple email and password authentication process students can access their finances.

Mobile Banking is very appealing to the audience we chose to market this application to as they can monitor deposits and review their transaction history. They can also transfer money from their current account to their savings account. Because of this, students will also find it easier to save money and be able to access these savings whenever they need to. Students don’t have a lot of money so being conscious of the way they handle their money is very important so incorporating a savings feature is very beneficial especially when students have to worry about paying bills, rent, college fees etc.

	


\chapter{Context}
\begin{itemize}
\item Provide a context for your project.
\item Set out the objectives of the project
\item Briefly list each chapter / section and provide a 1-2 line description of what each section contains.
\item List the resource URL (GitHub address) for the project and provide a brief list of the main elements at the URL.
\end{itemize}



\chapter{Methodology}

\section{Methodologies}
  
  Since this is a research project we have experimented with many different techniques to improve the creation of our application. We have utilised variations of extreme programming, 
  Agile and Scrum methodologies. 
  Agile Methodologies such as Scrum and Agile are tremendously important in the current technological sector which is why we have utilised the Agile Methodology during the creation of this application\cite{agile}. 
  Iterative development is a subject of much scrutiny in modern software companies as the sector is looking to move away from the rigid and constrained "Waterfall" development process\cite{agilewaterfall}. 
  
  
  \subsection{Agile Methodology}
  The Agile development method is much more suitable for developing applications as it focuses on creating a minimum viable product at the end of each development cycle. 
  The minimum viable product is a standalone part of the application that has been designed, coded and tested within a single sprint allowing for much more control over, 
  the development and evolution of the application. The application is evaluated at the end of every sprint and integrated into the full product. 
  Git branches have been very beneficial for the task of integrating our sprint products into the application. 
  
  \subsection{Scrum Methodology} 
  We have started as a two person team and have been joined by another member near the end of our development. The Scrum methodology which focuses on the interactions between 
  individuals rather than code and documentation has helped us co-operate as a team and integrate the newest member with our team. 
  This focus on interactions also means we have regularly met with our project supervisor who acted as our client for this application, helping us guide the 
  development process based on the products we were able to present to him. 
  The team had many online channels of communication such as Facebook Messenger, GitHub issues and email through which we could co-operate and solve problems together. 
  Outside of the regular meetings with the supervisor we have met regularly as a team, applying the stand up meetings of the mentioned methodologies during which 
  we could discuss any issues that we have encountered as well as present the current progress of our designated tasks. 
  At many of those meetings we would apply techniques similar to extreme programming into our sessions\cite{scrum}. 
  One person would be on the computer creating the code while the other was guiding him through the process and watch for any mistakes and errors. 
  This is one of the most successful techniques we have used during the creation of this project, it allowed for much faster and smoother code creation while 
  at the same time safeguarding from most errors that would not be spotted by one person.


\chapter{Technology Review}
About seven to ten pages.
\begin{itemize}
\item Describe each of the technologies you used at a conceptual level. Standards, Database Model (e.g. MongoDB, CouchDB), XMl, WSDL, JSON, JAXP.
\item Use references (IEEE format, e.g. [1]), Books, Papers, URLs (timestamp) – sources should be authoritative. 
\end{itemize}

\section{XML}
Here's some nicely formatted XML:
\begin{minted}{xml}
<this>
  <looks lookswhat="good">
    Good
  </looks>
</this>
\end{minted}

\chapter{System Design}
As many pages as needed.
\begin{itemize}
\item Architecture, UML etc. An overview of the different components of the system. Diagrams etc… Screen shots etc.
\end{itemize}

\begin{table}[h]
  \centering
  \begin{tabular}{x{2cm}p{3cm}}
    \toprule \\
    Column 1 & Column 2 \\
    \midrule \\
    Rows 2.1 & Row 2.2 \\
    \bottomrule
  \end{tabular}
  \caption{A table.}
  \label{table:mytable}
\end{table}

\chapter{System Evaluation}
As many pages as needed.
\begin{itemize}
\item Prove that your software is robust. How? Testing etc. 
\item Use performance benchmarks (space and time) if algorithmic.
\item Measure the outcomes / outputs of your system / software against the objectives from the Introduction.
\item Highlight any limitations or opportuni-ties in your approach or technologies used.
\end{itemize}

\chapter{Conclusion}
About three pages.

\begin{itemize}
\item Briefly summarise your context and ob-jectives (a few lines).
\item Highlight your findings from the evalua-tion section / chapter and any opportuni-ties identified.
\end{itemize}

